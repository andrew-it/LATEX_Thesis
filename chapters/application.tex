\chapter{Application}
\label{chap:application}

\begin{lstlisting}[caption={Class declaration in C}, label={lst:C_class_decl}]
/* ---------------- Shape.h ---------------- */

/* Shape's attributes... */ 
typedef struct {
    int16_t x; /* x-coordinate of Shape's position */
    int16_t y; /* y-coordinate of Shape's position */ 
} Shape;

/* Shape's operations (Shape's interface)... */
void Shape_ctor(Shape * const me, int16_t x, int16_t y); 
void Shape_moveBy(Shape * const me, int16_t dx, int16_t dy);

/* ---------------- Shape.c ---------------- */

/* constructor */
void Shape_ctor(Shape * const me, int16_t x, int16_t y) {
    me->x = x;
    me->y = y; 
}
/* move-by operation */
void Shape_moveBy(Shape * const me, int16_t dx, int16_t dy) {
    me->x += dx;
    me->y += dy; 
}

/* Creating objects */ 
int main() {
    Shape s1, s2; /* multiple instances of Shape */
    Shape_ctor(&s1, 0, 1);
    Shape_ctor(&s2, -1, 2); 
    Shape_moveBy(&s1, 2, -4);
}
\end{lstlisting}

\begin{lstlisting}[caption={Inheritance in C}, label=lst:C_inheritance]
#include "Shape.h"
/* ---------------- Rectangle.h ---------------- */

typedef struct {
    Shape super; /* <== inherits Shape */ 
    /* attributes added by this subclass... */ 
    uint16_t width;
    uint16_t height;
} Rectangle;

/* constructor */
void Rectangle_ctor(Rectangle * const me, 
                    int16_t x, int16_t y, 
                    uint16_t width, uint16_t height);

/* ---------------- Rectangle.c ---------------- */
void Rectangle_ctor(Rectangle * const me, int16_t x, int16_t y,
uint16_t width, uint16_t height) 
{
    /* first call superclass ctor */
    Shape_ctor(&me->super, x, y);

    /* next, you initialize the attributes added by this subclass... */ 
    me->width = width;
    me->height = height;
}

/* Creating objects */ 
int main() {
    Rectangle r1, r2;
    /* instantiate rectangles... */ 
    Rectangle_ctor(&r1, 0, 2, 10, 15); 
    Rectangle_ctor(&r2, -1, 3, 5, 8);
    /* re-use inherited function from the superclass Shape... */ 
    Shape_moveBy((Shape *)&r1, -2, 3);
    Shape_moveBy(&r2->super, 2, -1);
}
\end{lstlisting}


\begin{lstlisting}[caption={Shape virtual table}, label={lst:Shapt_virt_table}]
struct ShapeVtbl {
    uint32_t ( area* )(Shape * const me); 
    void ( draw* )(Shape * const me);
};
\end{lstlisting}


\begin{lstlisting}[caption={Adding virtual pointer to Shape class}, label={lst:Shapt_virt_pointer}]
struct ShapeVtbl; /* forward declaration */ 
typedef struct {
    struct ShapeVtbl const *vptr; /* <= Shape's vptr */
    int16_t x;
    int16_t y;
} Shape;
\end{lstlisting}


\begin{lstlisting}[caption={Defining the virtual table and initializing the virtual pointer}, label={lst:Shape_vtbl_vptr}]
/* Shape class implementation of its virtual functions... */ 
static uint32_t Shape_area_(Shape * const me);
static void Shape_draw_(Shape * const me);
/* constructor */
void Shape_ctor(Shape * const me, int16_t x, int16_t y) {
    static struct ShapeVtbl const vtbl = { 
        &Shape_area_,
        &Shape_draw_
    };
    me->vptr = &vtbl; /* "hook" the vptr to the vtbl */

    me->x = x;
    me->y = y; 
}
\end{lstlisting}


\begin{lstlisting}[caption={Defining purely virtual functions}, label={lst:Pure_virt}]
/* Shape class implementations of its virtual functions... */ 
static uint32_t Shape_area_(Shape * const me) {
    ASSERT(0); /* purely-virtual function should never be called */
    return 0; /* to avoid compiler warnings */ 
}
    
static void Shape_draw_(Shape * const me) {
    ASSERT(0); /* purely-virtual function should never be called */
}
\end{lstlisting}

\lstset{
  language=json,
  breaklines=true,
  frame=L,
  basicstyle=\footnotesize\ttfamily,
  morekeywords={version, language, patterns, dependencies, standard_types, standard_functions}
}

\begin{lstlisting}[caption={Patterns JSON file for C language}, label={lst:Json_patterns}]
{
  "version": 0.1,
  "language": "C",
  "patterns": {
    "STR_VAR": {
      "pattern": "\""
    },
    "CHAR_VAR": {
      "pattern": "'"
    },
    "LOOP": {
      "pattern": "for(INITIAL_STATE, WHILE_CONDITION, STEP_EXPR){BODY}"
    },
    "IF_CONDITION": {
      "pattern": "if(CONDITION){\nBODY\n}"
    },
    "ELSE_CONDITION": {
      "pattern": "else{\nBODY\n}"
    },
    "ELIF_CONDITION": {
      "pattern": "else if(CONDITION){BODY}"
    },
    "TERNARY": {
      "pattern": "CONDITION?TRUE:FALSE;"
    },
    "FUNC_DECL": {
      "signature_delimiter": ",",
      "pattern": "\nRET_TYPE FUNC_NAME(SIGNATURE){\nBODY\n}"
    },
    "ENTRY_POINT": {
      "signature_delimiter": ",",
      "pattern": "\nRET_TYPE FUNC_NAME(SIGNATURE){\nBODY\n}"
    },
    "FUNC_CALL": {
      "signature_delimiter": ",",
      "pattern": "FUNC_NAME(PARAMS);"
    },
    "SIG_PARAM": {
      "pattern": "RET_TYPE VAR_NAME"
    },
    "SIG_PARAM_POINTER": {
      "pattern": "RET_TYPE *VAR_NAME"
    },
    "SIG_PARAM_CONST_POINTER": {
      "pattern": "RET_TYPE *const VAR_NAME"
    },
    "VAR_DECL": {
      "pattern": "RET_TYPE VAR_NAME;"
    },
    "VAR_DECL_POINTER": {
      "pattern": "RET_TYPE *VAR_NAME;"
    },
    "VAR_ASSIGNMENT": {
      "pattern": "VAR_NAME = VALUE;"
    },
    "EMPTY_OPERATOR": {
      "operator": ";"
    },
    "DEPENDENCY": {
      "pattern": "#include <LIB>"
    },
    "VAL_STR": {
      "pattern": "\"VALUE\""
    },
    "VAL_INT": {
      "pattern": "VALUE"
    },
    "HEADER_FILE": {
      "pattern": "#ifndef TYPE_ALIAS_H\n#define TYPE_ALIAS_H\ntypedef struct {\nMEMBER_DECLARATIONS\n} TYPE_ALIAS; FUNCTION_PROT_DECLARATIONS\n#endif"
    },
    "CLASS_DECL": {
      "pattern": "#include \"TYPE_ALIAS.h\"\nFUNCTION_DECLARATIONS"
    },
    "FUNC_PROTOTYPE": {
      "signature_delimiter": ",",
      "pattern": "\nRET_TYPE FUNC_NAME(SIGNATURE);"
    },
    "FUNC_RETURN": {
      "pattern": "return VAR_NAME;"
    },
    "SYMBOLIC_SEQ": {
      "pattern": "VALUE"
    },
    "ONE_STRING_COMMENT": {
      "pattern": "// VALUE"
    }
  },
  "dependencies": {
    "stdio.h": ["printf", "scanf"],
    "sthlib.h": ["sth"]
  },
  "standard_types": {
    "STRING": "char*",
    "INT": "int",
    "REAL": "float",
    "CHAR": "char",
    "VOID": "void"
  },
  "standard_functions": {
    "console_out": {
      "name": "printf",
      "str": "%s",
      "int": "%d"
    },
    "console_in": {
      "name": "scanf",
      "str": "%s",
      "int": "%d"
    }
  }
}
\end{lstlisting}

\begin{lstlisting}[caption={Patterns JSON file for JavaScript language}, label={lst:Json_JS_patterns}]
{
  "version": 0.1,
  "language": "JavaScipt",
  "patterns": {
    "STR_VAR": {
      "pattern": "\""
    },
    "CHAR_VAR": {
      "pattern": "'"
    },
    "LOOP": {
      "pattern": "for(INITIAL_STATE, WHILE_CONDITION, STEP_EXPR){BODY}"
    },
    "IF_CONDITION": {
      "pattern": "if(CONDITION){\nBODY\n}"
    },
    "ELSE_CONDITION": {
      "pattern": "else{\nBODY\n}"
    },
    "ELIF_CONDITION": {
      "pattern": "else if(CONDITION){BODY}"
    },
    "TERNARY": {
      "pattern": "CONDITION?TRUE:FALSE;"
    },
    "FUNC_DECL": {
      "signature_delimiter": ",",
      "pattern": "function FUNC_NAME(SIGNATURE){BODY}"
    },
    "ENTRY_POINT": {
      "signature_delimiter": ",",
      "pattern": "BODY"
    },
    "FUNC_CALL": {
      "signature_delimiter": ",",
      "pattern": "FUNC_NAME(PARAMS)"
    },
    "SIG_PARAM": {
      "pattern": "VAR_NAME"
    },
    "SIG_PARAM_POINTER": {
      "pattern": "VAR_NAME"
    },
    "SIG_PARAM_CONST_POINTER": {
      "pattern": "VAR_NAME"
    },
    "VAR_DECL": {
      "pattern": "var VAR_NAME;"
    },
    "VAR_DECL_POINTER": {
      "pattern": "var VAR_NAME;"
    },
    "VAR_ASSIGNMENT": {
      "pattern": "VAR_NAME = VALUE;"
    },
    "EMPTY_OPERATOR": {
      "operator": ";"
    },
    "DEPENDENCY": {
      "pattern": "# LIB"
    },
    "VAL_STR": {
      "pattern": "\"VALUE\""
    },
    "VAL_INT": {
      "pattern": "VALUE"
    },
    "HEADER_FILE": {
      "pattern": "var TYPE_ALIAS = {MEMBER_DECLARATIONS\nFUNCTION_PROT_DECLARATIONS\n};"
    },
    "CLASS_DECL": {
      "pattern": "# TYPE_ALIAS | FUNCTION_DECLARATIONS"
    },
    "FUNC_PROTOTYPE": {
      "signature_delimiter": ",",
      "pattern": "# FUNC_NAME | SIGNATURE)"
    }
  },
  "dependencies": {},
  "standard_types": {
    "STRING": "",
    "INT": "",
    "REAL": "",
    "CHAR": "",
    "VOID": ""
  },
  "standard_functions": {
    "console_out": {
      "name": "console.log",
      "str": "%s",
      "int": "%d"
    },
    "console_in": {
      "name": "scanf",
      "str": "%s",
      "int": "%d"
    }
  }
}
\end{lstlisting}


\begin{lstlisting}[caption={C make file pattern}, label={lst:C_make}]
CC=g++
CFLAGS=-c -Wall
LDFLAGS=
SOURCES=LIST_OF_SOURCES
OBJECTS=$(SOURCES:.cpp=.o)
EXECUTABLE=EXECUTABLE_NAME

all: $(SOURCES) $(EXECUTABLE)
	
$(EXECUTABLE): $(OBJECTS) 
	$(CC) $(LDFLAGS) $(OBJECTS) -o $@

.cpp.o:
	$(CC) $(CFLAGS) $< -o $@
\end{lstlisting}
\lstset{
  language=C,
  basicstyle=\tiny,
  breaklines=true,
  numbers=none,
  frame=none,
  morekeywords={in, loop, end, unit, extend, override, Integer}
}
\begin{center}
\begin{tabular}{ | l | l | } \hline
SLang & Patterns \\ \hline
% CONDITION
\begin{lstlisting}
if CONDITION then
    // condition_true body
else
    // condition_false body
end
\end{lstlisting} & 
\begin{lstlisting}
if(CONDITION) {
    // condition_true body    
}
else {
    // condition_false body
}
\end{lstlisting} \\ \hline

% LOOP
\begin{lstlisting}
while ELEMENT in COLLECTION loop
    // loop body
end
\end{lstlisting} & 
\begin{lstlisting}
for(; ELEMENT != null; COLLECTION.next()){
    // loop body
    // where ELEMENT is sth like:
    // COLLECTION.getCurrentElement()
}
\end{lstlisting} \\ \hline

% CLASS
\begin{lstlisting}
unit Base
    x: Integer
    foo()
end
\end{lstlisting} & 
\begin{lstlisting}
typedef struct {
    int x;
} Base;
void Base_foo(Base* self){}
\end{lstlisting} \\ \hline

% INHERITANCE
\begin{lstlisting}
unit Derived extend Base, Base2
    y: Integer
    override foo()
end
\end{lstlisting} & 
\begin{lstlisting}
typedef struct {
    Base super;
    Base2 super2;
    int y;
} Derived;
Derived_foo(Derived* self){}
\end{lstlisting} \\ \hline

% NESTED FUNCTIONS
\begin{lstlisting}
foo(i: Integer) is
    boo(j: Integer, i: Integer) is 
        System.IO.print(i+j)
    end boo
    boo(1, i)
end foo
\end{lstlisting} & 
\begin{lstlisting}
void foo(int i) {
    void boo(int j, int i);
    boo(1, i);
}
void boo(int j, int i){
    printf("%d", i+j);
}
\end{lstlisting} \\ \hline
\end{tabular}
\end{center}